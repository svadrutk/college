% Options for packages loaded elsewhere
\PassOptionsToPackage{unicode}{hyperref}
\PassOptionsToPackage{hyphens}{url}
%
\documentclass[
]{article}
\usepackage{amsmath,amssymb}
\usepackage{iftex}
\ifPDFTeX
  \usepackage[T1]{fontenc}
  \usepackage[utf8]{inputenc}
  \usepackage{textcomp} % provide euro and other symbols
\else % if luatex or xetex
  \usepackage{unicode-math} % this also loads fontspec
  \defaultfontfeatures{Scale=MatchLowercase}
  \defaultfontfeatures[\rmfamily]{Ligatures=TeX,Scale=1}
\fi
\usepackage{lmodern}
\ifPDFTeX\else
  % xetex/luatex font selection
\fi
% Use upquote if available, for straight quotes in verbatim environments
\IfFileExists{upquote.sty}{\usepackage{upquote}}{}
\IfFileExists{microtype.sty}{% use microtype if available
  \usepackage[]{microtype}
  \UseMicrotypeSet[protrusion]{basicmath} % disable protrusion for tt fonts
}{}
\makeatletter
\@ifundefined{KOMAClassName}{% if non-KOMA class
  \IfFileExists{parskip.sty}{%
    \usepackage{parskip}
  }{% else
    \setlength{\parindent}{0pt}
    \setlength{\parskip}{6pt plus 2pt minus 1pt}}
}{% if KOMA class
  \KOMAoptions{parskip=half}}
\makeatother
\usepackage{xcolor}
\usepackage[margin=1in]{geometry}
\usepackage{color}
\usepackage{fancyvrb}
\newcommand{\VerbBar}{|}
\newcommand{\VERB}{\Verb[commandchars=\\\{\}]}
\DefineVerbatimEnvironment{Highlighting}{Verbatim}{commandchars=\\\{\}}
% Add ',fontsize=\small' for more characters per line
\usepackage{framed}
\definecolor{shadecolor}{RGB}{248,248,248}
\newenvironment{Shaded}{\begin{snugshade}}{\end{snugshade}}
\newcommand{\AlertTok}[1]{\textcolor[rgb]{0.94,0.16,0.16}{#1}}
\newcommand{\AnnotationTok}[1]{\textcolor[rgb]{0.56,0.35,0.01}{\textbf{\textit{#1}}}}
\newcommand{\AttributeTok}[1]{\textcolor[rgb]{0.13,0.29,0.53}{#1}}
\newcommand{\BaseNTok}[1]{\textcolor[rgb]{0.00,0.00,0.81}{#1}}
\newcommand{\BuiltInTok}[1]{#1}
\newcommand{\CharTok}[1]{\textcolor[rgb]{0.31,0.60,0.02}{#1}}
\newcommand{\CommentTok}[1]{\textcolor[rgb]{0.56,0.35,0.01}{\textit{#1}}}
\newcommand{\CommentVarTok}[1]{\textcolor[rgb]{0.56,0.35,0.01}{\textbf{\textit{#1}}}}
\newcommand{\ConstantTok}[1]{\textcolor[rgb]{0.56,0.35,0.01}{#1}}
\newcommand{\ControlFlowTok}[1]{\textcolor[rgb]{0.13,0.29,0.53}{\textbf{#1}}}
\newcommand{\DataTypeTok}[1]{\textcolor[rgb]{0.13,0.29,0.53}{#1}}
\newcommand{\DecValTok}[1]{\textcolor[rgb]{0.00,0.00,0.81}{#1}}
\newcommand{\DocumentationTok}[1]{\textcolor[rgb]{0.56,0.35,0.01}{\textbf{\textit{#1}}}}
\newcommand{\ErrorTok}[1]{\textcolor[rgb]{0.64,0.00,0.00}{\textbf{#1}}}
\newcommand{\ExtensionTok}[1]{#1}
\newcommand{\FloatTok}[1]{\textcolor[rgb]{0.00,0.00,0.81}{#1}}
\newcommand{\FunctionTok}[1]{\textcolor[rgb]{0.13,0.29,0.53}{\textbf{#1}}}
\newcommand{\ImportTok}[1]{#1}
\newcommand{\InformationTok}[1]{\textcolor[rgb]{0.56,0.35,0.01}{\textbf{\textit{#1}}}}
\newcommand{\KeywordTok}[1]{\textcolor[rgb]{0.13,0.29,0.53}{\textbf{#1}}}
\newcommand{\NormalTok}[1]{#1}
\newcommand{\OperatorTok}[1]{\textcolor[rgb]{0.81,0.36,0.00}{\textbf{#1}}}
\newcommand{\OtherTok}[1]{\textcolor[rgb]{0.56,0.35,0.01}{#1}}
\newcommand{\PreprocessorTok}[1]{\textcolor[rgb]{0.56,0.35,0.01}{\textit{#1}}}
\newcommand{\RegionMarkerTok}[1]{#1}
\newcommand{\SpecialCharTok}[1]{\textcolor[rgb]{0.81,0.36,0.00}{\textbf{#1}}}
\newcommand{\SpecialStringTok}[1]{\textcolor[rgb]{0.31,0.60,0.02}{#1}}
\newcommand{\StringTok}[1]{\textcolor[rgb]{0.31,0.60,0.02}{#1}}
\newcommand{\VariableTok}[1]{\textcolor[rgb]{0.00,0.00,0.00}{#1}}
\newcommand{\VerbatimStringTok}[1]{\textcolor[rgb]{0.31,0.60,0.02}{#1}}
\newcommand{\WarningTok}[1]{\textcolor[rgb]{0.56,0.35,0.01}{\textbf{\textit{#1}}}}
\usepackage{longtable,booktabs,array}
\usepackage{calc} % for calculating minipage widths
% Correct order of tables after \paragraph or \subparagraph
\usepackage{etoolbox}
\makeatletter
\patchcmd\longtable{\par}{\if@noskipsec\mbox{}\fi\par}{}{}
\makeatother
% Allow footnotes in longtable head/foot
\IfFileExists{footnotehyper.sty}{\usepackage{footnotehyper}}{\usepackage{footnote}}
\makesavenoteenv{longtable}
\usepackage{graphicx}
\makeatletter
\def\maxwidth{\ifdim\Gin@nat@width>\linewidth\linewidth\else\Gin@nat@width\fi}
\def\maxheight{\ifdim\Gin@nat@height>\textheight\textheight\else\Gin@nat@height\fi}
\makeatother
% Scale images if necessary, so that they will not overflow the page
% margins by default, and it is still possible to overwrite the defaults
% using explicit options in \includegraphics[width, height, ...]{}
\setkeys{Gin}{width=\maxwidth,height=\maxheight,keepaspectratio}
% Set default figure placement to htbp
\makeatletter
\def\fps@figure{htbp}
\makeatother
\setlength{\emergencystretch}{3em} % prevent overfull lines
\providecommand{\tightlist}{%
  \setlength{\itemsep}{0pt}\setlength{\parskip}{0pt}}
\setcounter{secnumdepth}{-\maxdimen} % remove section numbering
\ifLuaTeX
  \usepackage{selnolig}  % disable illegal ligatures
\fi
\IfFileExists{bookmark.sty}{\usepackage{bookmark}}{\usepackage{hyperref}}
\IfFileExists{xurl.sty}{\usepackage{xurl}}{} % add URL line breaks if available
\urlstyle{same}
\hypersetup{
  pdftitle={Stat 324 Homework \#1},
  pdfauthor={Svadrut Kukunooru},
  hidelinks,
  pdfcreator={LaTeX via pandoc}}

\title{Stat 324 Homework \#1}
\author{Svadrut Kukunooru}
\date{}

\begin{document}
\maketitle

*Submit your homework to Canvas by the due date and time. Email your
lecturer if you have extenuating circumstances and need to request an
extension.

*If an exercise asks you to use R, include a copy of the code and
output. Please edit your code and output to be only the relevant
portions.

*If a problem does not specify how to compute the answer, you many use
any appropriate method. I may ask you to use R or use manual
calculations on your exams, so practice accordingly.

*You must include an explanation and/or intermediate calculations for an
exercise to be complete.

*Be sure to submit the HWK1 Autograde Quiz which will give you 20 of
your 40 accuracy points.

*50 points total: 40 points accuracy, and 10 points completion

\hypertarget{basics-of-statistics-and-summarizing-data-graphically-i}{%
\subsection{Basics of Statistics and Summarizing Data Graphically
(I)}\label{basics-of-statistics-and-summarizing-data-graphically-i}}

\textbf{Exercise 1}. A number of individuals are interested in the
proportion of citizens within a county who will vote to use tax money to
upgrade a professional baseball stadium in the upcoming vote. Consider
the following methods:

The \textbf{Baseball Team Owner} surveyed 8,000 people attending one of
the baseball games held in the stadium. Seventy eight percent (78\%) of
respondents said they supported the use of tax money to upgrade the
stadium.

The \textbf{Pollster} generated 1,000 random numbers between 1-52,661
(number of county voters in last election) and surveyed the 1,000
citizens who corresponded to those numbers on the voting roll. Forty
three percent (43\%) of respondents said they supported the use of tax
money to upgrade the stadium.

\begin{quote}
\begin{enumerate}
\def\labelenumi{\alph{enumi}.}
\tightlist
\item
  What is the population of interest? What is the parameter of interest?
  Will this parameter ever be calculated?
\end{enumerate}
\end{quote}

The population of interest is the U.S. citizens and attendees of a
baseball game held in the stadium. The parameter of interest is the
proportion of citizens within a county will vote to use tax money to
upgrade a professional baseball stadium in the upcoming vote.

\begin{quote}
\begin{enumerate}
\def\labelenumi{\alph{enumi}.}
\setcounter{enumi}{1}
\tightlist
\item
  What were the sample sizes used and statistics calculated from those
  samples? Are these simple random samples from the population of
  interest?
\end{enumerate}
\end{quote}

The sample size used was the 1,000 citizens randomly selected using a
random number generator, as well as the 8,000 people attending a
baseball game. The statistics calculated from these samples are the
percentage of people who said they would support using tax money to
upgrade the baseball stadium. The pollster method uses a simple random
sample from the population of interest because they use a random number
generator, but the baseball team owner uses people who attend a certain
baseball game, which is not a simple random sample because the people in
the sample are not equally likely to be picked.

\begin{quote}
\begin{enumerate}
\def\labelenumi{\alph{enumi}.}
\setcounter{enumi}{2}
\tightlist
\item
  The baseball team owner claims that the survey done at the baseball
  stadium will better predict the voting outcome because the sample size
  was much larger. What is your response?
\end{enumerate}
\end{quote}

The survey done at the baseball stadium will not better predict the
voting outcome because the sample sise only consists of people who
attend a baseball game, which may be biased because they may be bigger
fans than the rest of the citizens of upgrading the stadium they attend.

\vspace{1cm}

\newpage

\textbf{Exercise 2:} After manufacture, computer disks are tested for
errors. The table below summarizes the number of errors detected on each
of the 100 disks produced in a day.

\begin{longtable}[]{@{}ll@{}}
\toprule\noalign{}
Number of Defects & Number of Disks \\
\midrule\noalign{}
\endhead
\bottomrule\noalign{}
\endlastfoot
0 & 42 \\
1 & 30 \\
2 & 16 \\
3 & 7 \\
4 & 5 \\
\end{longtable}

\begin{quote}
\begin{enumerate}
\def\labelenumi{\alph{enumi}.}
\tightlist
\item
  Describe the type of data that is being recorded about the sample of
  100 disks, being as specific as possible.
\end{enumerate}
\end{quote}

The data being recorded about the sample of 100 disks is the amount of
defects on the disk.

\begin{quote}
\begin{enumerate}
\def\labelenumi{\alph{enumi}.}
\setcounter{enumi}{1}
\tightlist
\item
  Code for a frequency histogram showing the frequency for number of
  errors on the 100 disks is given below.
\end{enumerate}
\end{quote}

\begin{quote}
\begin{quote}
bi. Knit the document and confirm that the histogram displays in the
knitted file.
\end{quote}
\end{quote}

\begin{Shaded}
\begin{Highlighting}[]
\NormalTok{error.data}\OtherTok{=}\FunctionTok{c}\NormalTok{(}\FunctionTok{rep}\NormalTok{(}\DecValTok{0}\NormalTok{,}\DecValTok{42}\NormalTok{), }\FunctionTok{rep}\NormalTok{(}\DecValTok{1}\NormalTok{,}\DecValTok{30}\NormalTok{), }\FunctionTok{rep}\NormalTok{(}\DecValTok{2}\NormalTok{,}\DecValTok{16}\NormalTok{), }\FunctionTok{rep}\NormalTok{(}\DecValTok{3}\NormalTok{,}\DecValTok{7}\NormalTok{), }\FunctionTok{rep}\NormalTok{(}\DecValTok{4}\NormalTok{, }\DecValTok{5}\NormalTok{))}
\FunctionTok{hist}\NormalTok{(error.data, }\AttributeTok{breaks=}\FunctionTok{c}\NormalTok{(}\FunctionTok{seq}\NormalTok{(}\AttributeTok{from=}\SpecialCharTok{{-}}\FloatTok{0.5}\NormalTok{, }\FloatTok{4.5}\NormalTok{, }\AttributeTok{by=}\DecValTok{1}\NormalTok{)), }
     \AttributeTok{xlab=}\StringTok{"Defects"}\NormalTok{, }\AttributeTok{main=}\StringTok{"Number of Defects"}\NormalTok{, }
     \AttributeTok{labels=}\ConstantTok{TRUE}\NormalTok{, }\AttributeTok{ylim=}\FunctionTok{c}\NormalTok{(}\DecValTok{0}\NormalTok{,}\DecValTok{50}\NormalTok{))}
\end{Highlighting}
\end{Shaded}

\includegraphics{HWK1_324_SS_files/figure-latex/unnamed-chunk-1-1.pdf}

\begin{quote}
\begin{quote}
bii. Describe what the rep() function does in this code chunk.
\end{quote}
\end{quote}

The rep() function creates vectors for each column in the histogram;
e.g., the first \texttt{rep()}, \texttt{rep(0,42)} makes the column for
the amount of disks that had zero defects, 42.

\begin{quote}
\begin{quote}
biii. Describe how this breaks command affects the histogram's
appearance in this code chunk. How would the histogram look different if
the breaks() command was not included?
\end{quote}
\end{quote}

The histogram would look different in that it would use ranges rather
than discrete numbers. It would be split into ranges like 0-1, 1-2, 2-3,
etc.

\begin{quote}
\begin{quote}
biv. Describe how setting ylim=c(0,30) instead of ylim=c(0,50) would
change the histogram's appearance. Which value for ylim is perferable
for clear communication of the data?
\end{quote}
\end{quote}

Setting \texttt{ylim=c(0,30)} would limit the y-axis to a maximum of 30.
50 would be a preferable value for clear communication of the data.

\vspace{1cm}

\end{document}
