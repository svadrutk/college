\documentclass[letterpaper]{report}
\setlength\parindent{0pt}
\usepackage[utf8]{inputenc}
\usepackage[T1]{fontenc}
\usepackage{textcomp}
\usepackage{amsmath, amssymb}

\title{Lecture 7: Row Echelon Forms}
\author{Svadrut Kukunooru}
\date{\today}


\begin{document}
\begin{titlepage}
    \maketitle
\end{titlepage}

A matrix  is said to be in \textit{reduced row echelon form} (RREF) if: 
\begin{enumerate}
    \item Rows of zeroes appear at bottom of matrix
    \item First nonzero entry at each row is one
    \item Nonzero entries are placed into an echelon form
    \item All entries above leading ones are zero 
\end{enumerate}
If a matrix is not in RREF, we can make it into one by using elementary row operations, e.g.
\begin{itemize}
    \item  Exchange two rows
    \item Multiply row by scalar
    \item Add one row to another
\end{itemize}
\subsection{Example Problem}%
\label{sub:Example Problem}
Simplify the following matrix to RREF: 
\[
\begin{bmatrix} 
    -1 & 2 & 5 \\
    2 & -1 & 6 \\ 
    2 & -2 & 7
\end{bmatrix} 
.\] 
First, we can subtract the 2nd and 3rd row by two copies of row one, e.g.
\[
\begin{bmatrix} 
    1 & -2 & 5 \\
    0 & 3 & -4 \\
    0 & 2 & -3
\end{bmatrix} 
.\] 
Then, subtract the 2nd row by two copies of the 3rd row:
\[
\begin{bmatrix} 
    1 & -2 & 5 \\
    0 & 1 & -1 \\
    0 & 2 & -3
\end{bmatrix} 
.\] 

Subtract the third row by two copies of the second row: 

\[
\begin{bmatrix} 
    1 & -2 & 5 \\
    0 & 1 & -1 \\
    0 & 0 & 1
\end{bmatrix} 
.\] 
Some more manipulations: 
\[
\begin{bmatrix} 
    1 & -2 & 5 \\
    0 & 1 & 0 \\
    0 & 0 & 1
\end{bmatrix} 
\rightarrow
\fbox{$
\begin{bmatrix} 
    1 & 0 & 0 \\
    0 & 1 & 0 \\
    0 & 0 & 1
\end{bmatrix}$}
.\] \newpage
The procedure is this: 
\begin{enumerate}
    \item  Make the leading term of 1st row 1 by scaling
    \item Kill the 1st entries of all rows below
    \item Make the leading form of 2nd row 1 by scaling
    \item Kill the leading terms below the second leading 1
    \item Repeat the above until you get row echelon form
    \item Kill the entries below leading 1's by going bottom-up
\end{enumerate}
\textbf{DEF}: Two matrices are said to be row equivalent if one is obtained from another using elementary row operations. \\ \\ 
\textbf{THEOREM}: Any matrix is row equivalent to one in RREF, which is unique (rank? I can't read). A matrix can be row equivalent to multiple row echelon forms, e.g.
\[
\begin{bmatrix} 
    1 & 1 \\
    0 & 1
\end{bmatrix} 
\text{ and }
\begin{bmatrix} 
    1 & 0 \\
    0 & 1
\end{bmatrix} 
.\] 
\subsection{Example Problem}%
\label{sub:Example Problem}
Reduce the following matrix to RREF: 
\[
\begin{bmatrix} 
    1 & 2 & 0 & 3 & 5 \\
    0 & 0 & 1 & 2 & 1 \\
    0 & 0 & 0 & 0 & 1
\end{bmatrix} 
.\] 
The only reason this matrix isn't already in RREF is that the 5 and 1 are nonzero entries above the 3rd leading 1. Therefore, 
\[
\fbox{$
\begin{bmatrix} 
    1 & 2 & 0 & 3 & 0 \\
    0 & 0 & 1 & 2 & 0 \\
    0 & 0 & 0 & 0 & 1
\end{bmatrix}$}
.\] 
\end{document}
