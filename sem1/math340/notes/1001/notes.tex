\documentclass[letterpaper]{report}

\setlength\parindent{0pt}
\usepackage[utf8]{inputenc}
\usepackage[T1]{fontenc}
\usepackage{textcomp}
\usepackage{amsmath, amssymb}
\title{Lecture 012: Introduction to Determinants}
\author{Svadrut Kukunooru}
\date{\today}

\begin{document}
\begin{titlepage}
    \maketitle
\end{titlepage}
A determinant defines a function $\text{det} : M_{n\times n} \to \mathbb{R}$. Recall that in a 2x2 case, 
\[
    det\left(
    \begin{bmatrix} 
        a & b \\
        c & d
\end{bmatrix}\right)  = ad - bc
\] 
\textbf{SIGNED AREA}: Suppose you want to integrate the sin function from $-\pi$ to $\pi$: 
\[
    \int_{-\pi}^{\pi} \sin(x)
\] 
The answer to this is zero, since the are aon the left side of the y-axis negates the area on the right side of the y-axis. POINT: It is natural to assign area a sign. 
\\ \\ 
Given vectors $\vec{v_1}$ and $\vec{v_2}$ on $\mathbb{R}$ we denote by $P(v_1, v_2)$ the \textit{parallelogram} defined by $v_1$ and $v_2$. Note that the signed area of $P(v_1,v_2)$ is 0 if $v_1$ and $v_2$ are parallel/colinear. It is positive if $v_1$ rotates counterclockwise towards $v_2$ in $P(v_1, v_2)$, and negative if $v_1$ rotates clockwise towards $v_2$.  \\ \\ 
\textbf{THEOREM}: If $A$ is a $2\times 2$ matrix with row vectors $r_1$ and $r_2$, then the determinant of A is the signed area of $P(r_1,r_2)$ For example say you have a matrix 
\[
A = 
\begin{bmatrix} 
    3 & 1 \\
    2 & 3
\end{bmatrix} 
\begin{bmatrix} 
r_1 \\ r_2
\end{bmatrix} 
\] How do you find the area of the parallelogram?
Most poeple use the \textbf{SLIDE METHOD}, where 
\end{document}
