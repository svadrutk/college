\documentclass[letterpaper]{report}

\setlength\parindent{0pt}
\usepackage[utf8]{inputenc}
\usepackage[T1]{fontenc}
\usepackage{textcomp}
\usepackage{amsmath, amssymb}
\title{Lecture 013}
\author{Svadrut Kukunooru}
\date{\today}

\begin{document}
\begin{titlepage}
    \maketitle
\end{titlepage}
Recall that last timed we showed that if $A$ is a $2\times 2$ matrix w/ row vectors $\vec{r_1}, \vec{r_2}, \ldots$, or \[A = \begin{bmatrix} 
r_1 \\ r_2 \\ \ldots 
\end{bmatrix} \]
then the determinant of $A$ is the signed area of $P(r_1, r_2)$ (the parallelogram formed by the two vectors). 
\\ \\ 
\textbf{DEFINITION}: for an $n \times n$ matrix $A$, det($A$) is defined to be the signed volume of $P(r_1, r_2, \ldots r_n)$ (the parallelotope; basically a 3-D version of a parallelogram defined by the row vectors $r_1, r_2, \ldots r_n$). 
\\ \\ 
\textbf{NOT TESTED ON EXAM, BUT GOOD TO KNOW}: The sign of det($A$) reflects whether the matrix function $f_A: \mathbb{R}^n \implies \mathbb{R}^n$ preserves the orientation of $\mathbb{R}^n$ orientation. \\ \\ 
For $R_1$, the orientation is defined as either positive or negative. For $R_2$, the orientation is defiend as either counterclockwise or clockwise. In $R_3$, the orientation is defined as either right-handed or left-handed. \\ \\ 
\\ \\ 
We can conclude that whenever $E$ is an elementary matrix, 
\[
    \text{det}(EA) = \text{det}(E) \cdot \text{det}(A)
\] 
This rule continues to hold for higher dimensional parallelotopes. METHOD: perform elementary row operations to simplify $A$ to one of the base cases, keep track of the determinants of the elementary matrices. 
\\ \\ 
Recall for an $n \times  n$ matrix $A$, 
\[
    \text{RREF}(A) = 
    \begin{cases}
             I_n\ \Rightarrow \text{(A is invertible)} 
             \\
             \text{has a row of zeroes $\Rightarrow$ (A is not invertible)}
    \end{cases}
\] 
\\ \\ 
\textbf{EXAMPLE}: Compute the determinant of $A$ if 
\[
A = 
\begin{bmatrix} 
    1 & 1 & 2 \\
    2 & 1 &3 \\
    3 & 2 & 1
\end{bmatrix} 
\] 
\[
\begin{bmatrix} 
    1 & 1  &2 \\
    2 & 1 & 3 \\
    3 & 2 & 1 
\end{bmatrix} \to 
\begin{bmatrix} 
    1 & 1 & 2 \\
    0 & -1 & -1 \\
    0 & -1 & -5 
\end{bmatrix} \to 
\begin{bmatrix} 
    1 & 1 & 2 \\
    0 & -1 & -1 \\
    0 & 0 & -4
\end{bmatrix} \to
\] 
\[
\begin{bmatrix} 
    1 & 1 & 2 \\
    0 & 1 & 1 \\
    0 & 0 & -4
\end{bmatrix} \to 
\begin{bmatrix} 
    -1 & 1 & 2 \\
    0 & 1 & 1 \\
    0 & 0 & -4
\end{bmatrix} \to 
\begin{bmatrix} 
    -1 & 1 & 2 \\
    0 & 1 & 1 \\
    0 & 0 & 1
\end{bmatrix} \to 
I_3
\] \\ \\ 
Therefore, the determinant is 1, since the determinant of an identity matrix must always be 1. 
\end{document}
