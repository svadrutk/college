\documentclass[letterpaper]{report}

\setlength\parindent{0pt}
\usepackage[utf8]{inputenc}
\usepackage[T1]{fontenc}
\usepackage{textcomp}
\usepackage{amsmath, amssymb}
\title{Lecture 007}
\author{Svadrut Kukunooru}
\date{\today}

\begin{document}
\begin{titlepage}
    \maketitle
\end{titlepage}

\textbf{ELECTRONIC SWITCHES} are the foundation upon which all modern computers are built. They are switches controlled electrically instead of mechanical action (e.g. a light switch requires someone to physically move the switch). \\ \\ 
\textbf{TRANSISTORS} are electronic switches made of a semiconductor material, such as a metal oxide. (MOS stands for metal oxide semiconductor) \\ \\
Transistors use a voltage at their "gate" to control electronic switches (e.g. they require a certain amount of voltage to open their gate) \\ \\ 

N-type and P-type MOS transistors are both used to form CMOS logic gates. 

To create a NOT gate, we can simply use both an n-type and p-type transistor. N-Type doesn't allow 0's and P-type doesn't allow 1's. The logic symbol for a NOT gate is a horizontal line, sideways triangle, circle, horizontalline. \\ \\ 

MAKE SURE TO CHECK WHAT THE SYMBOLS ARE FOR N- AND P-TYPE RESISTORS \\ \\ 

We went over NAND and NOR, but they really aren't that hard, just a negation of AND/OR \\ \\ 

Gates with more than 2 inputs are formed from 2 input gates, but multiple ones. Also, any truth table can be implemented with and, or and not gates. 


\end{document}
