\documentclass[letterpaper]{report}

\setlength\parindent{0pt}
\usepackage[utf8]{inputenc}
\usepackage[T1]{fontenc}
\usepackage{textcomp}
\usepackage{amsmath, amssymb}
\title{Lecture 005: The Big Ideas}
\author{Svadrut Kukunooru}
\date{\today}

\begin{document}
\begin{titlepage}
    \maketitle
\end{titlepage}

\textbf{In THEORY}, all computers, if give enough \textit{time}and \textit{memory}, are capable of computing anything that can be computed. 
\\ \\ 
The A (short for automatic)-Machine, creating by 1940s scientist Alan Turing, is a mathematical model of computation. It, in short, defines a very simple abstract computer. 

\begin{enumerate}
    \item Reads/writes symbols on an infinite memory tape (refer to the theory above)
    \item Transitions to a different state based on 
        \begin{itemize}
            \item The current state of progress
            \item A finite table of instructions
            \item Current symbol on the tape
        \end{itemize} 
\end{enumerate}
Every computation can be performed by some Turing Machine. 

\[
    A,B \to  T_{add} \to A + B
.\] 

\[
    A,B \to T_{mul} \to  A \times B
.\] 
However, it is important to note that NOT EVERYTHING is computable! \\ \\ 
\rule{40}{2} \\ \\ 
A \textbf{UNIVERSAL TURING MACHINE} is a Turing machine that can implement all other turing machines. Its input includes data AND a program.
\[
T_{add}, T_{mul}, \left{A,B,C\right} \to U \to A \cross (B+C)
\]
Most modern machines today are universal Turing machines, such as our phones, laptops, etc. In practice, solving problems with computers involves constraints, such as \textbf{time}, \textbf{cost}, and \textbf{power}.  \\\\
\rule{40}{2} \\ \\ 
\textbf{ABSTRACTION} manages complexity by focusing on some attributes over others. Solving a problem using a computer is a systematic sequence of transformations between levels of abstraction. 
\newpage
The sequence is as follows:  \\ \\ 
\textbf{PROBLEM STATEMENT}
\begin{itemize}
    \item High-level description in a natural language
    \item Usually ambiguous, imprecise, or maybe incorrect
\end{itemize}
\textbf{ALGORITHMS \& DATA STRUCTURES}
\begin{itemize}
    \item Step-by-Step procedures and information organization
    \item Unambiguous, computable, finite specifications
\end{itemize}
\textbf{PROGRAM}
\begin{itemize}
    \item Statements that are organized into modules (e.g. functions, methods, etc.)
    \item Variables that are designed to contain data organized into arrays, structures, objects, etc.
    \item High-level, structured, logical
\end{itemize}
\textbf{INSTRUCTION SET ARCHITECTURE (ISA)}
\begin{itemize}
    \item Instructions the computer can perform and data types the computer can understand
    \item Low-level; tied to particular types of machines (e.g. x86, ARM, amd64, etc.)
\end{itemize}
\textbf{MICROARCHITECTURE}
\begin{itemize}
    \item Detailed processor implementation (e.g. i3,i5, ryzen 5, etc.)
\end{itemize}
\textbf{CIRCUITS}
\begin{itemize}
    \item Logic gates \& low-level components
\end{itemize}
\textbf{DEVICES}
\begin{itemize}
    \item 100's of millions of electronic switches turning on and off billions of times a second
    \item Properties of maeterials, manufacturability
\end{itemize}
Note that there are many options at each level of abstraction; different problems, programming languages, devices, processors, wires, etc. 
\end{document}
