\documentclass[letterpaper]{report}

\setlength\parindent{0pt}
\usepackage[utf8]{inputenc}
\usepackage[T1]{fontenc}
\usepackage{textcomp}
\usepackage{amsmath, amssymb}
\title{Lecture 014}
\author{Svadrut Kukunooru}
\date{\today}

\begin{document}
\begin{titlepage}
    \maketitle
\end{titlepage}
   \textbf{EXAMPLE}: Compute the determinant of the following matrix $A$: 
   \[
   A = 
   \begin{bmatrix} 
       3 & -1 & 2 \\
       4 & 5 & 6 \\
       7 & 1 & 2 
   \end{bmatrix} \to 
   \begin{bmatrix} 
       4 & 5 & 6 \\
       3 & -1 & 2 \\
       7 & 1 & 2 
   \end{bmatrix} \to
   \begin{bmatrix} 
       1 & 6 & 4 \\
       3 & -1 & 2 \\
       7 & 1 & 2
   \end{bmatrix} \to 
   \begin{bmatrix} 
       1 & 6 & 4 \\
       0 & -19 & -10 \\
       0 & -41 & -26
   \end{bmatrix} \to \ldots
   \] 
   \[
       \to \fbox{84}
   \] 
   We will begin with a new way of computing determinants, the cofactor expansion formula.  \\ \\ 
   \textbf{DEF}: Let $A$ by a square matrix. The minor $M_{ij}$ is the matrix obtained by deleting the $i$th row and $j$th column of $A$. For example, 
   \[
   A = 
   \begin{bmatrix} 
       3 & -1 & 2 \\
       4 & 5 & 6 \\
       7 & 1 & 2 
   \end{bmatrix}, 
   M_{1.1} = 
   \begin{bmatrix} 
       5 & 6 \\
       1 & 2
   \end{bmatrix} , 
   M_{2.2} = 
   \begin{bmatrix} 
       3 & 2 \\
       7 & 2 
   \end{bmatrix},
   M_{2.3} = 
   \begin{bmatrix} 
       3 & -1 \\
       7 & 1 
   \end{bmatrix} 
   \] 
   \textbf{DEF}: The cofactor at $a_{ij}$ is defined as as 
   \[
       A_{ij} = (-1)^{i+j} \text{det}(M_{ij})
   \] 
   To compute the determinant, take any row or column, take out all the indices (i,j) in this row/columnm then sum up $a_{ij}A_{ij}$. e.g. on a 3x3 matrix, if we compute $\text{det}(A)$ by exapnding from the first row, then we do 
   \[
       a_{11}A_{11} + a_{12}A_{12} + a_{13}A_{13}
   \] 
\textbf{EXAMPLE}: Compute the determinant of the matrix using cofactor expansion. 
\[
\begin{bmatrix} 
    3 & -1 & 2 \\
    4 & 5 & 6 \\
    7 & 1 & 2
\end{bmatrix} 
\] 
\[
    a_{11}A_{11} + a_{12}A_{12} + a_{13}A_{13}
\] 
\[
    = 3\ \text{det}\begin{bmatrix} 
        5 & 6 \\
        1 & 2
    \end{bmatrix} - (-1) \ \text{det}
    \begin{bmatrix} 
        4 & 6 \\
        7 & 2
    \end{bmatrix} + 2 \ \text{det}
    \begin{bmatrix} 
        4 &5 \\
        7 & 1
    \end{bmatrix} 
\] 
\[
    = 3 \cdot 4 - 34 +2(-31) = 12 - 34 - 62  = \fbox{84}
\] 
\textbf{EXAMPLE}: Compute the determinant of the following matrix: 
\[
\begin{bmatrix} 
    4 & 1 & 3 \\
2 & 3 & 0  \\
1 & 3 & 2 
\end{bmatrix} \to
4\ \text{det}
\begin{bmatrix} 
    3 & 0 \\
    3 & 2 
\end{bmatrix} - 1\ \text{det}
\begin{bmatrix} 
    2 & 0 \\
    1 & 2 
\end{bmatrix} + 3\ \text{det}
\begin{bmatrix} 
    2 & 3 \\
    1 & 3 
\end{bmatrix}  \] 
\[
= 24-4+9 = \fbox{29}
\] 
\textbf{THEOREM}: If two matrices $A$ and $A'$ are row equivalent, that means $Ax = 0$ and $A'x = 0$ have the same solutions. Additionally, a homogenous equation $Ax = 0$ is always consistent and has infinitely many solutions if and only if it has a free variable. Finally, if $A$ is a square matrix, then the following are equivalent: 
\[
    Ax = 0 \text{ has at least one free variable }
\] 
\[
    \text{RREF}(A) \text{ has a row of zeroes }
\] 
\[
    A \text{ is not invertible }
\] 
\[
    \text{det }(A) = 0 
\] 
(basically, if you know one of these statements is true, all of them are)
\end{document}
