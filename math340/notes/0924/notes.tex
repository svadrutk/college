\documentclass[letterpaper]{report}

\setlength\parindent{0pt}
\usepackage[utf8]{inputenc}
\usepackage[T1]{fontenc}
\usepackage{textcomp}
\usepackage{amsmath, amssymb}
\title{Lec 009: RREFs, Continued}
\author{Svadrut Kukunooru}
\date{\today}

\begin{document}
\begin{titlepage}
    \maketitle
\end{titlepage}
\textbf{RECAP}: An RREF looks like this: 
\[
\begin{bmatrix} 
    1 & x & x \\
    0 & 1 & x \\
    0 & 0 & 1
\end{bmatrix} 
.\] 
The diagonal of ones is called an \textbf{ECHELON}. 
Note that even though the diagonal of ones might be interrupted, the matrix will still be in RREF, such as: 
\[
\begin{bmatrix} 
    0 & 1 & 0 & 0 & 0 \\
    0 & 0 & 1 & 0 & 0 \\
    0 & 0 & 0 & 0 & 0 \\
    0 & 0 & 0 & 0 & 1
\end{bmatrix} 
.\] 
\subsection{EXAMPLE PROBLEM}%
\label{sub:EXAMPLE PROBLEM}
Solve this system of equations: 
\begin{equation}
    \begin{cases}
        x + 2y + 3z = 9 \\ 
        2x - y + z = 8 \\
        3x - z = 3
    \end{cases}
\end{equation}
\begin{equation}
    \begin{cases}
        x + 2y + 3z = 9 \\
        -5y -5z = -10 \\
        -6y-10z = -24
    \end{cases}
\end{equation}
\begin{equation}
    \begin{cases}
        x +  2y + 3z = 9 \\
        y + z = 2 \\
        z = 3 
    \end{cases}
\end{equation}
\[\fbox{$
        x = 2,
        y = -1,
        z = 3
$}\]

Next, convert the following matrix to RREF: 
\[
\begin{bmatrix} 
    1 & 2 & 3 \\ 
    2 & -1 & 1 \\
    3 & 0 & -1
\end{bmatrix} 
\begin{bmatrix} 
9 \\ 10 \\ -24
\end{bmatrix}  \implies
\begin{bmatrix} 
    1 & 2 & 3 \\
    0 & -5 & -5 \\
    0 & -6 & -10
\end{bmatrix} 
\begin{bmatrix} 
9 \\ 10 \\ -24
\end{bmatrix} 
.\] 
\[
\implies 
\begin{bmatrix} 
    1 & 2 & 3 \\
    0 & 1 & 1 \\
    0 & 0 & 1
\end{bmatrix} 
\begin{bmatrix} 
9 \\ 2 \\ 3
\end{bmatrix} \implies
\begin{bmatrix} 
    1 & 2 & 3 \\
    0 & 1 & 0 \\
    0 & 0 & 1
\end{bmatrix} 
\begin{bmatrix} 
9 \\ -1 \\ 3
\end{bmatrix} 
\implies
\fbox{$
\begin{bmatrix} 
    1 & 0 & 0 & 2 \\
    0 & 1 & 0 & -1 \\
    0 & 0 & 1 & 3
\end{bmatrix} $}
.\] 
\rule{40}{2} \\ \\ 
You can see that solving a system of linear equations in the form $A\vec{x} = \vec{b}$ is the same thing as reducing $[A | \vec{b}]$ to its RREF. 

\subsection{EXAMPLE PROBLEM}%
\label{sub:EXAMPLE PROBLEM}
\begin{equation}
    \begin{cases}
        x + y + z + w = 0 \\
        x + w = 0 \\
        x + 2y + z  = 0 
    \end{cases}
\end{equation}
What are the solutions to this system? 
HINT: The RREF is 
\[
\begin{bmatrix} 
    1 & 0 & 0 & 1 \\
    0 & 1 & 0 & -1 \\
    0 & 0 & 1 & 1 
\end{bmatrix} 
\begin{bmatrix} 
0 \\ 0 \\ 0
\end{bmatrix} 
.\] 
Convert the RREF to a system of equations: 
\begin{equation}
    \begin{cases}
        x + w = 0 \\
        y - w = 0 \\
        z + w = 0 
    \end{cases}
\end{equation}
A variable is called a \textbf{FREE VARIABLE} if the corresponding column in RREF does not have a leading one. Other wise, it is a \textbf{LEADING VARIABLE}. In the above example $w$ is the free variable and $x,y,z$ are the leading variables. Set parameters for free variables, then express the leading variables in terms of these parameters. 

\end{document}
